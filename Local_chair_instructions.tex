%%%%%%%%%%%%%%%%%%%%%%%%%%%%%%%%%%%%%%%%%%%%%%%%%%%%%%%%%%%%%%%%%%%%%%%%%%%%%%%
%%
%%          $Id: Rulebook.tex $
%%    author(s): RoboCupAtHome Technical Committee(s)
%%  description: introduction to Local chair
%%
%%%%%%%%%%%%%%%%%%%%%%%%%%%%%%%%%%%%%%%%%%%%%%%%%%%%%%%%%%%%%%%%%%%%%%%%%%%%%%%
\documentclass[11pt, twoside, openright, a4paper, chapterprefix]{article}
% \usepackage[inner=2.5cm, outer=2.5cm, top=4cm, bottom=4cm]{geometry}

%%% PACKAGES %%%%%%%%%%%%%%%%%%%%%%%%%%%%%%%%%%%%%%%%%%%%%%%%%%%%%%%%%%%%%%%%%%
\usepackage{graphicx}
% \input{./setup_tex/packages.tex}
% \input{./setup_tex/config.tex}
% \input{./setup_tex/styling.tex}
% \input{./setup_tex/active_version.tex}
% \input{./setup_tex/abbrevix.tex}
% \input{./setup_tex/macros.tex}
% \input{./setup_tex/macros_score_sheets.tex}
% \input{./setup_tex/macros_open_demonstrations.tex}
% \input{./setup_tex/macros_leagues.tex}

% \graphicspath{{\YEAR/}{./images/}}
\graphicspath{{./images/}}

% \makeindex                               
% \makeabbex                             

% \newcommand{\sectionbreak}{\clearpage}
% \newcommand{\subsectionbreak}{\clearpage}

\begin{document}

%  titlepage
\begin{titlepage}
    \begin{center}
      {
        \includegraphics[width=.25\textwidth]{images/logo_RoboCupFed.jpg}
        \hfill
        \includegraphics[width=.23\textwidth]{images/logo_rcbrazilhome.png}\\
        [1.23ex]
      }
      \vspace{2.7 cm}
      \hrulefill\par
      {%
        \vspace*{.27cm}
        \Huge{RoboCup Brazil Open}\\[1.23ex]
        \Huge{RoboCup@Home}\\[1.23ex]
        \Large Instuctions for Local Chair 2023\\[2ex]
      }
      \hrulefill\par
      \vfill
      ~~ Last Build Date: \today \quad 
      \vfill
    \end{center}
  \end{titlepage}
%   \clearpage


% \input{./titlepage}
% \pagestyle{empty}
% %% %%%%%%%%%%%%%%%%%%%%%%%%%%%%%%%%%%%%%%%%%%%%%%%%%%%%%%%%%%%%%%%%%%%%%%%%%%%
%%
%%          $Id: acknowledgments.tex 404 2013-02-15 08:51:20Z sugiura $
%%    author(s): RoboCupAtHome Technical Committee(s)
%%  description: Acknowledgments for the RoboCupAtHome RuleBook
%%
%% %%%%%%%%%%%%%%%%%%%%%%%%%%%%%%%%%%%%%%%%%%%%%%%%%%%%%%%%%%%%%%%%%%%%%%%%%%%


\section*{About this Rulebook}  
This is the official rulebook of the \AtHome{} competition for the RoboCup Brazil Open Competition \YEAR.

It has been written based on the official RoboCup@home rulebook (\url{https://github.com/RoboCupAtHome/RuleBook/}). 

\section*{Committee Members for \YEAR}
\begin{itemize}
    \item \textbf{Executive Committee:} \\ Fagner de Assis Moura Pimentel $<fpimentel@fei.edu.br>$
    \item \textbf{Technical and Organizing Committee:} \\ 
        Ana Patricia Magalhães (UNEB/UNIFACS) \\
        Jardel dos Santos Dyonisio (FURG) \\
        Kristofer Kappel (UFPEL) \\
        Gabriela Bassegio (FEI) \\
        Gustavo Fardo Armênio (UTFPR) \\
        José Rafael Rebêlo Teles (UFG) \\
        Rhayna Christiani Vasconcelos Marques Casado (USP)
\end{itemize}

\section*{Acknowledgments}
\label{sec:acknowledgments}

We would like to thank all the people who contributed to the \AtHome{} league with their feedback and comments.

\vfill

\section*{License}

\doclicenseThis


% \pagestyle{empty}
% \input{./example_skills.tex}
% \clearpage

% \pagestyle{empty}
% \setcounter{tocdepth}{1}
% \tableofcontents
\clearpage

% \pagestyle{plain}

% \renewcommand{\chapter}{Lecture}


\section{Arena}

\begin{tabular}{ p{0.3\linewidth} p{0.01\linewidth} p{0.6\linewidth}}
    \textbf{Área}                               & : & Entre $80m^2$ e $100m^2$. \\
    \textbf{Estilo}                          & : & Padrão do pais/região sede. \\
    \textbf{Decoração}                   & : & De acordo com os costumes da região sede. \\
    \textbf{Composição mínima}  & : & Sala de estar, escritório, dormitório e cozinha.
\end{tabular}


\subsection{Paredes}

\begin{tabular}{ p{0.3\linewidth} p{0.01\linewidth} p{0.6\linewidth}}
    \textbf{Altura mínima}           & : & 60 cm \\
    \textbf{Altura máxima}           & : & Não deve atrapalhar a visão do público. \\
    \textbf{Decoração}                   & : & As paredes podem conter cartaz ou quadro. 
\end{tabular}

\subsection{Piso}

\begin{tabular}{ p{0.3\linewidth} p{0.01\linewidth} p{0.6\linewidth}}
    \textbf{Tipo}                     & : & Com alguma aderência para evitar que os robôs derrapem. \\
    \textbf{Cor/textura}     & : & Mesma cor ou textura das parede \\
    \textbf{Degraus}              & : & Não deve haver degraus. \\
    \textbf{Carpete}              & : & Não deve haver carpete. 
\end{tabular}

\subsection{Portas}

\begin{tabular}{ p{0.3\linewidth} p{0.01\linewidth} p{0.6\linewidth}}
    \textbf{Quantidade}   & : & Pelo menos duas portas. Entrada e saída da arena. A Porta de entrada deve abrir para dentro, enqanto a porta de saída deve abrir para fora. \\
    \textbf{Maçanetas}     & : & As maçanetas devem ser de preferencia retanguralres. Não usar maçanetas redondas. \\
\end{tabular}

\subsection{Móveis}

\begin{tabular}{ p{0.3\linewidth} p{0.01\linewidth} p{0.6\linewidth}}
    \textbf{Tipo}       & : & Típicos do país/região sede. \\
\end{tabular}

\begin{description}
    \item [1] Mesa de jantar pequena com duas a quatro cadeiras.
    \item [1] Sofá com dois ou tres lugares. 
    \item [2] Cadeiras ou poltronas 
    \item [1] Estante para televisão.
    \item [1] Televisão com controle remoto.
    \item [1] Armário ou prateleira (com alguns livros no interior) 
    \item [1] Uma geladeira (Não deve ter menos de 120 cm)
    \item [2] Vazos com plantas.
    \item [1] Máquina de lavar louça.
    \item [1] Cama.
    \item [1] Mesa de escritório com cadeira.
    \item [2] Vazos de lixo com 20 cm (mínimo) de diametro na abertura. Devem possuir tampas removíveis com alças rígidas.
\end{description}

\section{Objetos da arena}

\subsection{Utensílios de mesa}
\begin{description}
    \item [1] Conjunto de pratos (minímo 3 itens)
    \item [1] Conjunto de tigelas (minímo 3 itens)
    \item [1] Conjunto de xícaras (minímo 3 itens)
    \item [1] Pacote de guardanapo
    \item [1] Conjunto de Talheres (Garfo, faca e colher).
    \item [10] Sacos de lixo de plástico (Tamanho médio). 
    \item [4] Sacolas de compras: Leves e com alças rigidas.
    \item [2] Bandejas: Bandeja ou cesto, destinado à manipulação com as duas mãos.
    \item [1] Objeto derramável: cujo conteúdo pode ser derramado (por exemplo, jarro). 
    \item [1] Objeto Pesado: peso entre 1,0kg e 1,5kg (por exemplo, garrafa de água). 
    \item [1] Objeto minúsculo: leve, não maior que 5 cm (por exemplo, saquinho de chá). 
    \item [1] Objeto frágil: fácil de quebrar (por exemplo, ovo).
    \item [1] Um objeto deformável: flexível que pode aparecer em diferentes formas (por exemplo, tecido).
   
\end{description}

\subsection{Objetos para reconhecimento e manipulação}

Os objetos são apresentados aos competidores em forma de kits.

\begin{tabular}{ p{0.3\linewidth} p{0.01\linewidth} p{0.6\linewidth}}
    \textbf{Manipulação}   & : & todos os objetos devem ser pasiveis de serem manipulados com apenas uma mão e pesar no máximo 1.5KG. \\
    \textbf{Quantidade}     & : & A quantidade de kits é definida pelo número de equipes participantes. 1 kit para a arena e 1 kit para cada 3 equipes participantes. \\
    \textbf{Definição}       & : & Os Itens são definidos durante a competição (Warm Up). \\
    \textbf{Itens}                & : & Os kits devem ter entre 15 e 20 itens. Podem conter, por exemplo, os elementos a seguir. 
\end{tabular}

\resizebox{\textwidth}{!}{
    \begin{tabular}{|c|c|c|c|c|c|}
        \hline
        Drinks & Cleaning supplies & Pantry items & Fruits & Snacks & Cutlery\\
        \hline
        Coconut Water   & Cleaner  & Corn Flakes       & Apple   & Cookie       & Spoon \\
        Coke                      & Cloth       & English Sauce	 & Banana & Pringles   & Fork\\
        Guarana               & Sponge     & Mustard              & Orange & ~                   & Knife\\
        Ice Tea	               & ~                 & Tomato Sauce   & Kiwi      & ~                   & Plate\\ 
        Tonic                   & ~                 & Tuna Can            & ~             & ~                   & Bowl\\
        Water                   & ~                 & Corn Flakes     & ~             & ~                   & Mug\\
    \hline
    \end{tabular}
}

Obs: Um representante da organização local deve acompanhar o coordenador da categoria a um supermercado local para adiquirir os itens.

\section{Outros materiais}

\begin{description}
    \item [2] Cronômetros.
    \item [8] Suportes para poster.
    \item [1] Sistema de som com pelo menos uma lapela para ligar no robô.    
    \item [1] Monitor para apresentação de placar e informaçõe para o público.
\end{description}



\end{document}
