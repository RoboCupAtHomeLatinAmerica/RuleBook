\section{Open Challenge}
\label{sec:rules:openchallenge}

On the first two competition days after the regular test blocks ended, there will be an opportunity for teams to present an open challenge in which teams can demonstrate their novel research and approaches.

\subsection{Procedure}
\label{sec:rules:ocprocedure}
\begin{enumerate}
	\item \textbf{Participation:} Teams have to announce whether they want to perform an open challenge to the \abb{OC} during \SetupDays{}.
	\item \textbf{Time:} Each team gets a 10 minute time slot of which 8 minutes are for presenting and 2 minutes for questions by the audience.
	\item \textbf{Arena Changes:} The team can rearrange the arena when their time slot starts but all changes need to be reverted as soon as their time slot ends.
	\item \textbf{Focus:} While the demonstrations are intended to share research insights, we still want to see robots performing. Do not turn the open challenge into an academic lecture.
	\item \textbf{Leagues:} Ideally, all \AtHome{} leagues' open challenges will be scheduled consecutively, to give the opportunity to watch all open challenges. In case there are more than 12 participants across leagues, each league will hold their open challenge concurrently.
	\item \textbf{Award:} Participating teams are eligible to receive the \OCAward{} (see \ref{sec:introduction:ocaward}).
\end{enumerate}
