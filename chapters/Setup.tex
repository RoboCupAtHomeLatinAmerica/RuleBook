\chapter{Setup Days}
\label{chap:setupdays}
The first days at a \RoboCup\AtHome{} competition before the tests start are the \SetupDays{}. This time is used by teams to assemble and test their robots and adjust to the local scenario. To foster knowledge exchange between teams, a \PS{} is held. To ensure safety and compliance with the rules, a \RobotInspection{} is conducted.

\section{General Setup}
\label{sec:setupdays:general}
Depending on the overall \RoboCup{} schedule, the \SetupDays{} last for one or two days.

\begin{itemize}
	\item \textbf{Start:} They start when the venue opens for the first time.
	\item \textbf{Intention:} Teams setup their team area and robots.
	\item \textbf{Tables:} The \abb{LOC} will setup and randomly assign team tables.
	\item \textbf{Arena:} The \Arenas{} are available to all teams of the respective league. The \abb{OC} may schedule special test or mapping slots in which arena access is limited. Note, that furnishing may not be complete yet.
	\item \textbf{Objects:} The delegation of \abb{EC}, \abb{TC}, \abb{OC} and \abb{LOC} will buy the objects (see~\ref{sec:rules:scenario:objects}). Note, that the objects may not be available at all times and not from the beginning.
\end{itemize}

\section{Poster Session}
\label{sec:setupdays:postersession}
The \PS{} is for teams to present their research to the \AtHome{} community. Before the session a \WelcomeReception{} is held. The time before and after the \PS{} is for teams to exchange knowledge and to get to know each other.
\begin{itemize}
	\item \textbf{Time:} The \PS{} is held in the evening of the last setup day.
	\item \textbf{Place:} It takes place in the \Arena{} and/or in the team area.
	\item \textbf{Welcome Reception:} Time for teams to gather for the \PS{}. Snacks and beverages (beers, sodas, etc.) are served.
	\item \textbf{Organization:} It is the responsibility of the \abb{OC} and the \abb{LOC} to organize catering and location. This includes:
		\begin{itemize}
			\item Poster stands for each team or alternatives to present the posters.
			\item Snacks and drinks.
			\item Inviting officials, sponsors, \abb{LOC}, and \RCF{} trustees to the event.
		\end{itemize}
	\item \textbf{Poster Presentation:} Each team gives a short presentation of their poster.
	\item \textbf{Discussion:} Afterwards, teams are free to look at the posters, ask questions and discuss the presentations.
\end{itemize}

\subsection*{Poster Presentation}
\label{sec:setupdays:posterpresentation}
\begin{itemize}
	\item \textbf{Time:} Each team has a maximum of five minutes to give a short presentation of their poster.
	\item \textbf{Evaluation:} The posters are evaluated by a jury consisting of one member (preferable the team leader) of each team. The evaluation should be based on the presentation, as well as, any questions and discussions.
	\item \textbf{Criteria:} For each of the following evaluation criteria, a maximum of 10 points is given per jury member:
	\begin{itemize}
		\item Novelty and scientific contribution.
		\item Relevance for \RoboCup\AtHome{}.
		\item Presentation (Quality of poster, presentation style, and discussion).
	\end{itemize}
	\item \textbf{Score:} The points given by each jury member are scaled to obtain a maximum of 100 points. The total score for each team is the mean of the jury member scores. To neglect outliers, the N best and worst scores are left out. The points are added to a team's \SONE{} score:
	$$
	score=\frac{\sum \text{team-leader-score}}{\text{number-of-teams}}
	$$
	\item \textbf{Sheet collection:} Evaluation sheets are collected by the \OC{} at an announced time.
\end{itemize}

\section{Robot Inspection}
\label{sec:setupdays:inspection}
Passing the \RobotInspection{} is necessary for a robot to participate in any test.

\begin{itemize}
	\item \textbf{Schedule:} The \RobotInspection{} is held during the last day of the \SetupDays{}. A team order is announced by the \OC{} beforehand.
	\item \textbf{Procedure:} The inspection starts, like a regular test, with the opening of the entrance door. The robot needs to enter the \Arena{} and drive to a designated inspection point. On command (team's choice) the robot leaves through the exit door.
	\item \textbf{Inspectors:} The robots are inspected by the \TC{}.
	\item \textbf{Checked aspects:} It is checked if the robots comply with the rules (see~\ref{chap:rules}), checking in particular:
	\begin{itemize}
		\item Emergency button(s).
		\item Collision avoidance. An inspector steps in front of the robot.
		\item Voice of the robot. It must be loud and clear.
		\item Custom containers (bowl, tray, etc.).
		\item External devices.
		\item Alternative Human-Robot interfaces.
		\item Robot speed and dimension.
		\item Start button.
		\item Other safety issues (duct tape, hanging cables, sharp edges etc.).
	\end{itemize}
	\item \textbf{Re-Inspection:} If the robot is not approved, it is the responsibility of the team to get the approval later. This means, retrying directly after the regular \RobotInspection{} schedule or asking the \abb{TC} to be inspected at a later time.
	\item \textbf{Time Limit:} No strict time limit is given since approval of external devices can take time. But, inactive robots and robots moving too slowly or not towards the inspection point are removed quickly.
	\item \textbf{Accompanying Team Member:} Each robot is accompanied by only one team member (team leader is advised).
	\item \textbf{OC instructions (at least 2h before the Robot Inspection):}
	\begin{itemize}
		\item Announce the entry and exit doors.
		\item Announce the location of the inspection point.
	\end{itemize}
\end{itemize}


% Local Variables:
% TeX-master: "Rulebook"
% End:
