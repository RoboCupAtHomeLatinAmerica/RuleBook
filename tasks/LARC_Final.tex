\chapter{Finals}
\label{chap:finals}

The competition ends with the Finals on the last day.
% where the two teams with the highest total score compete.
The \iterm{Finals} are conducted as a final themed demonstration.
%To avoid logistical issues during the last day of the competition, the \iterm{Finals} are divided into two sets of demonstrations: the Bronze Competition and the RoboCup @Home Grand Finale.
%The Bronze Competition is a set of demonstrations that are carried out before the RoboCup @home Grand Finale. Here, all the leagues run in parallel, with the fourth and third highest scored teams competing for the bronze.
%Finally, the two teams with the highest score in each League present their demonstrations in a serialized manner during the RoboCup @Home Grand Finale.
% Even though each league has its own first, second and third place, 
The \iterm{Finals} are meant to show the best of the league to the jury members as well as the audience.
% and, thus, warrants a single schedule slot.

\section{Structure and Theme}

% The \iterm{Finals} are a demonstration of achieving an objective that is pre-selected by the TC/EC. 
% These objectives are chosen as a type of yearly theme of the competition, and to provide a baseline for the juries (not to mention the audience) to state which team is the winner.

The presentation must be conducted in two stages:
The first stage should be a pitch for the robot project in 5 minutes. 
The second stage must be a demonstration of the real robot performing a specific task determined by the TC/EC in 10 minutes.

The objectives for the league for this year are:
% The robot helps a person that has had a small accident in their home.
% The robot helps a person in preparing dinner.
% The robot helps an elderly person with their routine.
The robot must deal with a laundry routine. The team can strategically focus on all steps of the laundry (collecting clothes, separating clothes, putting clothes in the washing machine, putting cleaning products, removing clothes from the machine, putting clothes to dry, ironing, folding, and storing clothes), or can focus on just a few of these steps. The robot can perform the task alone or helping the human being collaboratively. It will be up to the judges to decide which solution was best sold and executed.



% \begin{itemize}
%     \item \emph{OPL/DSPL}: 
%     \item \emph{SSPL}: The robot monitors a person while they are going about their day and reacts appropriately if it notices any unusual events.
% \end{itemize}


The teams are expected to provide a demonstration that is telling a story which includes achieving the objective. The teams can choose freely how to achieve it, which includes choosing the participants, what items to use, the methods employed, etc. The juries, as explained later, will reward elegance and difficulty.

As it can be seen, the objectives are open enough that a story can be told around them which can include additional objectives that the team wants their robot to also solve. Thus, the teams are welcome to include in their demonstration any additional tasks to be solved, which can serve as a type of forum where they can present their own research. The innovation and success of these tasks will also be used as part of the score. In this regard, it is expected that teams present the scientific and technical contributions they submitted in \iterm{team description paper}.

In addition, teams may provide a printed document to the jury (max 1 page) that summarizes the demonstrated robot capabilities and contributions. However, teams are discouraged to provide any material that would distract from their demonstration.

Story-telling is an important factor, so it is recommended to spend the least amount of time using the microphone to explain the demonstration and let the demonstration speak for itself.

Finally, teams must be ready for unforeseen events and continue the presentation. The Jury should not wait for the team to prepare again if a problem occurs with the robot.


\section{Evaluating Juries for Final Demonstrations}
For the final, the max score will be 2000pts.
The \iterm{Finals} are evaluated by two juries, here described.

\begin{enumerate}
\item\textbf{League-internal jury:} The league-internal jury is formed by people with a background in robotics. They are appointed by the Executive Committee. The evaluation of the league-internal jury is based on the following criteria and represent 60\% of the score:
  \begin{compactenum}
  \item Efficacy/elegance of the solution.
  \item Innovation/contribution to the league of the additional tasks solved.
  \item Difficulty of the overall demonstration.
  \end{compactenum}

\item \textbf{League-external jury:} The league-external jury consists of people without a background in robotics. They are appointed by the Executive Committee. The evaluation of the league-external jury is based on the following criteria and represent 40\% of the score:
  \begin{compactenum}
  \item Originality and presentation (story-telling is to be rewarded).
  \item Relevance/usefulness to everyday life.
  \item Elegance/success of overall demonstration.
  \end{compactenum}
\end{enumerate}

% \section{Scoring}
% The final score and ranking are determined by the jury evaluations and by the previous performance (in Stages I and II) of the team, in the following manner:

% \begin{enumerate}
%   \item The influence of the league-internal jury to the final ranking is \SI{25}{\percent}.
%   \item The influence of the league-external jury to the final ranking is \SI{25}{\percent}.
%   \item The influence of the total sum of points scored by the team in Stage I and II is \SI{50}{\percent}.
% \end{enumerate}

%These demonstrations are carried out in a serialized fashion, one League performing after another in one \Arena{}.


\subsection{Task}
The procedure for the demonstration and the timing of slots is as follows:
\OpenDemonstrationTask{fifteen}{fifteen}

\OpenDemonstrationChanges

%% %%%%%%%%%%%%%%%%%%%%%%%%
% \section{Final Ranking and Winner}

% There will be an award for 1st, 2nd and 3rd place.% of each league.

% The winner of the competition is the team that gets the highest ranking in the \iterm{Finals}.

% The second place will be the team that got the second-highest ranking in the \iterm{Finals}.

% The third place will be the team with the highest score that did not made it to the \iterm{Finals}.

% Additional certificates would be granted if:

% \begin{enumerate}
%   \item If the number of teams in the league is above 11, a certificate will be awarded to the 4th ranked team.
%   \item If the number of teams in the league is above 14, a certificate will be awarded to the 5th ranked team.
% \end{enumerate}

\subsection{Score sheets}

\clearpage
\begin{center}
\textbf{Internal Jury:}
\end{center} 

\paragraph{Instructions:}
\begin{compactenum}
\item Please watch all demonstrations carefully.
\item At the end of the demonstration you may ask questions.
\item After demonstration and questions, please fill in the evaluation sheet below.
\item You may use the space below \enquote{Remarks:} to take notes for yourself.
\item Enter your your name (\enquote{referee name}) on top of the sheet.
\item Sign the form using the \enquote{Referee} slot at the bottom.
\end{compactenum}

\paragraph{Criteria:}
\begin{compactitem}
\item Scientific contribution
\item Contribution to @Home
\item Relevance for @Home / Novelty of approaches
\item Presentation and performance in the finals
\end{compactitem}



\paragraph{Evaluation sheet}
\begin{center}

\resizebox{\textwidth}{!}{  
  \begin{tabular}{|l|c|c|c|c|}
    \hline
    \multirow{3}{*}{Team name}
    &  Efficacy of solution         &  Elegance of                & Innovation/contribution & Difficulty/Success of  \\
    &  to main objective${}^\star$  &  solution to main objective & of additional tasks     & overall demonstration  \\
    &  (0-10)                       &  (0-10)                     &  (0-10)                 &  (0-10)  \\
    \hline
    & & & & \\\hline
    & & & & \\\hline
    & & & & \\\hline
    \hline
  \end{tabular}\\
}
${}^\star$ Penalize if the time is over.
\end{center}

Obs:


\clearpage
\begin{center}
\textbf{External Jury:}
\end{center} 

\paragraph{Instructions:}
\begin{compactenum}
\item Please watch all demonstrations carefully.
\item At the end of the demonstration you may ask questions.
\item At the end of all presentations, evaluate the teams comparatively.
\item After demonstration and questions, please fill in the evaluation sheet below.
\item You may use the space below \enquote{Remarks:} to take notes for yourself.
\item Enter your your name (\enquote{referee name}) on top of the sheet.
\item Sign the form using the \enquote{Referee} slot at the bottom.
\end{compactenum}


\paragraph{Evaluation sheet}
\begin{center}

\resizebox{\textwidth}{!}{  
\begin{tabular}{|l|c|c|c|c|}
  \hline
  \multirow{3}{*}{Team name}
  & Originality and        &  Relevance/usefulness to  &  Elegance of             &  Success of           \\
  & presentation${}^\star$ &  everyday life            &  overall demonstration   &  overall demonstration \\
  & (0-10)                 &  (0-10)                   &  (0-10)                  &  (0-10)                \\
  \hline
   & & & & \\\hline
   & & & & \\\hline
   & & & & \\\hline
  \hline
\end{tabular}\\
}
${}^\star$ Story telling is to be rewarded.
\end{center}



Obs:
