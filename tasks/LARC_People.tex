\section{Personal Recognition}
\label{test:personal-recognition}

\subsection*{Description}
An Operator is introduced to the robot, which needs to learn what the Operator looks like. Once the robot has gathered enough information about the Operator, the Operator mixes within a crowd and the robot needs to find the Operator. Once the robot has found its Operator, it must explain how it must state information about the Operator.

\noindent \textbf{Main goal:}
The robot has to identify the Operator within a crowd and state information about the Operator and the crowd.

\noindent \textbf{Optional goal:}
Identify the operator by your name.

\subsection*{Focus}
This task focuses on
\textit{people detection},
\textit{people recognition},
\textit{pose recognition} and
\textit{Human-robot interaction}.

\subsection*{Setup}
\begin{itemize}[nosep]	
	\item \textbf{Locations:} 
	\begin{itemize}
		\item This task takes place inside the \Arena{}.
		\item The robot will starts at a designated starting position. 
	\end{itemize}
	\item \textbf{People:} 
	\begin{itemize}
		\item A "professional" operator is selected by the TC to test the robot. This person may a different in each run. 
		\item The operator must be fluent in English. 
		\item The operator name will be drawn from a list of common English names (see section \ref{sec:rules:scenario:names}).
		\item The number of people in the crowd will be drawn. There will be a minimum of three people and a maximum of ten.
		\item The operator will be facing forward and the other people will not be facing backwards.
		\item The crowd will be located behind the robot at a distance between 2 and 3 meters apart.
	\end{itemize}
	\item \textbf{Furniture:} 
	\begin{itemize}
		\item All furniture are in their predefined locations.
	\end{itemize}
    \item \textbf{Objects:} 
    \begin{itemize}
		\item All objects are in their predefined locations.
	\end{itemize}
\end{itemize}

\subsection*{Procedure}
	\begin{enumerate}[nosep]
		\item The referee requests the team to move the robot to the start location.
		\item The referee gives the start signal and starts the timer.
		\item The team leaves the area after the start signal.		
		\item The referee follows the robot ready to press the emergency stop button.
		
		\item The team is allowed to instruct the operator until the referees start the time.
		\item The robot waits for the "professional" operator at the starting position.
		\item The robot has to memorize the operator. During this phase, the robot may instruct the operator to follow a certain setup procedure.
		\item \textbf{Optionally}, the robot may ask the operator for his/her name.
		Once the robot states it has finished memorizing the operator, it must wait for a Start Command.
		\item The operator walks around and blends into the crowd
		\item After the time elapses, the robot must turn about 180°, approach to the crowd and start looking for the operator.
		\item \textit{Optionally}, once the crowd has been located, the robot must greet the operator (navigation or with the manipulator).
% 		and state pose (sitting, standing, rising arms, etc.).
		\item Finally, robot must tell the size of the crowd .i.e. how many people there is in the crowd.
		

	\end{enumerate}

\subsection*{Additional rules and remarks}
\begin{itemize}[nosep]
	\item This test is not concerned with audio and voice recognition. Therefore, the start command may also be given by a single key press.
	\item The robot needs to wait for at least 1 min before the operator appears in front of the robot. During this waiting time the team is not allowed to touch the robot.
	\item If a person from the audience (severely) interferes with the robot in a way that makes it impossible to solve the task, the team may repeat the test immediately.
	\item The robot interacts with the operator, not the team. That is, the team is not allowed to instruct the operator.
    \item The robot needs to save an image log with boundbox.
\end{itemize}

\subsection*{Instructions:}
\subsubsection*{To Referee}

The referee needs to:
\begin{itemize}
	\item Check safe operation of the robot; the robot needs to be stopped immediately if a person is going to be touched by the robot.
	\item Choose operator name randomly.
\end{itemize}

\subsubsection*{To OC}
The OC needs to:
\begin{itemize}
    \item \textbf{2 hours before the test:} Select the "professional" operator.
    \item \textbf{2 hours before the test:} Select the crowd.
\end{itemize}

\newpage
\subsection*{Score sheet}

\begin{table}[h]
	\begin{tabular}{m{0.85\linewidth} c}
		\textbf{Maximum time}: & 5 minutes \\
	\end{tabular}
\end{table}

\begin{scorelist}

	\scoreheading{Regular Rewards}	
	
	\scoreitem[1]{400}{Correctly Identify the operator (boundbox) .}
	\scoreitem[1]{600}{Correctly State crowd's size (boundbox for each person).}

	\scoreheading{Bonus Rewards}
	\scoreitem[1]{60}{Approach or point at the operator (with navigation or manipulator).}
	\scoreitem[1]{40}{Identify the operator by his name. (boundbox with his name)}

	\scoreheading{Regular Penalties}
	\penaltyitem[10]{1}{False positive will remove a true positive score.}


\end{scorelist}

\clearpage
