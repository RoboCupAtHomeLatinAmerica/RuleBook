\subsection{Standard Platform Leagues}
\label{sec:rules:robotappearance_spl}
For Robots competing in a \SPL{}, modifications and alterations to the robots are strictly forbidden. This includes, but is not limited to, attaching, connecting, plugging, gluing, and taping components into and onto the robot, as well as, modifying or altering the robot structure. Not complying with this rule, leads to an immediate disqualification and penalization of the team (see~\ref{sec:rules:penaltiesbonuses}).

Robots are allowed to \enquote{wear} clothes, have stickers (e.g., a sticker exhibiting the logo of a sponsor), and be painted as long as they are compliant with section \ref{sec:rules:robotappearance}.

\subsubsection{DSPL Modifications}
\label{sec:rules:mountingbracket}
In the \DSPL{}, some modifications to the \HSR{} are allowed. An official \MountingBracket{} is provided by Toyota for the \HSR{}. Any laptop fitting inside the \MountingBracket{} can be used as additional on board computing. Furthermore, teams are allowed to attach the following devices to the robot or the laptop in the \MountingBracket{}:
\begin{enumerate}
	\item \textbf{Audio:} USB audio output device, e.g. USB-powered speaker, possibly with sound card.
	\item \textbf{Wi-Fi Adapter:} USB-powered IEEE 802.11ac (or newer) compliant device.
	\item \textbf{Ethernet Switch:} USB-powered IEEE 802.3ab (or newer) compliant device.
\end{enumerate}

\noindent A maximum of three such devices can be attached, they cannot increase the robot's dimension.