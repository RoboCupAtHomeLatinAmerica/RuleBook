%%%%%%%%%%%%%%%%%%%%%%%%%%%%%%%%%%%%%%%%%%%%%%%%%%%%%%%%%%%%%%%%%%%%%%%%%%%%%%%
%%
%%          $Id: Rulebook.tex $
%%    author(s): RoboCupAtHome Technical Committee(s)
%%  description: introduction to Local chair
%%
%%%%%%%%%%%%%%%%%%%%%%%%%%%%%%%%%%%%%%%%%%%%%%%%%%%%%%%%%%%%%%%%%%%%%%%%%%%%%%%
\documentclass[11pt, twoside, openright, a4paper, chapterprefix]{article}
% \usepackage[inner=2.5cm, outer=2.5cm, top=4cm, bottom=4cm]{geometry}

%%% PACKAGES %%%%%%%%%%%%%%%%%%%%%%%%%%%%%%%%%%%%%%%%%%%%%%%%%%%%%%%%%%%%%%%%%%
\usepackage{graphicx}
% \input{./setup_tex/packages.tex}
% \input{./setup_tex/config.tex}
% \input{./setup_tex/styling.tex}
% \input{./setup_tex/active_version.tex}
% \input{./setup_tex/abbrevix.tex}
% \input{./setup_tex/macros.tex}
% \input{./setup_tex/macros_score_sheets.tex}
% \input{./setup_tex/macros_open_demonstrations.tex}
% \input{./setup_tex/macros_leagues.tex}

% \graphicspath{{\YEAR/}{./images/}}
\graphicspath{{./images/}}

% \makeindex                               
% \makeabbex                             

% \newcommand{\sectionbreak}{\clearpage}
% \newcommand{\subsectionbreak}{\clearpage}

\begin{document}

%  titlepage
\begin{titlepage}
    \begin{center}
      {
        \includegraphics[width=.25\textwidth]{images/logo_RoboCupFed.jpg}
        \hfill
        \includegraphics[width=.23\textwidth]{images/logo_rcbrazilhome.png}\\
        [1.23ex]
      }
      \vspace{2.7 cm}
      \hrulefill\par
      {%
        \vspace*{.27cm}
        \Huge{RoboCup Brazil Open}\\[1.23ex]
        \Huge{RoboCup@Home}\\[1.23ex]
        \Large Instuctions for Local Chair 2023\\[2ex]
      }
      \hrulefill\par
      \vfill
      ~~ Last Build Date: \today \quad 
      \vfill
    \end{center}
  \end{titlepage}
%   \clearpage


% \input{./titlepage}
% \pagestyle{empty}
% %% %%%%%%%%%%%%%%%%%%%%%%%%%%%%%%%%%%%%%%%%%%%%%%%%%%%%%%%%%%%%%%%%%%%%%%%%%%%
%%
%%          $Id: acknowledgments.tex 404 2013-02-15 08:51:20Z sugiura $
%%    author(s): RoboCupAtHome Technical Committee(s)
%%  description: Acknowledgments for the RoboCupAtHome RuleBook
%%
%% %%%%%%%%%%%%%%%%%%%%%%%%%%%%%%%%%%%%%%%%%%%%%%%%%%%%%%%%%%%%%%%%%%%%%%%%%%%


\section*{About this Rulebook}  
This is the official rulebook of the \AtHome{} competition for the RoboCup Brazil Open Competition \YEAR.

It has been written based on the official RoboCup@home rulebook (\url{https://github.com/RoboCupAtHome/RuleBook/}). 

\section*{Committee Members for \YEAR}
\begin{itemize}
    \item \textbf{Executive Committee:} \\ Fagner de Assis Moura Pimentel $<fpimentel@fei.edu.br>$
    \item \textbf{Technical and Organizing Committee:} \\ 
        Ana Patricia Magalhães (UNEB/UNIFACS) \\
        Jardel dos Santos Dyonisio (FURG) \\
        Kristofer Kappel (UFPEL) \\
        Gabriela Bassegio (FEI) \\
        Gustavo Fardo Armênio (UTFPR) \\
        José Rafael Rebêlo Teles (UFG) \\
        Rhayna Christiani Vasconcelos Marques Casado (USP)
\end{itemize}

\section*{Acknowledgments}
\label{sec:acknowledgments}

We would like to thank all the people who contributed to the \AtHome{} league with their feedback and comments.

\vfill

\section*{License}

\doclicenseThis


% \pagestyle{empty}
% \input{./example_skills.tex}
% \clearpage

% \pagestyle{empty}
% \setcounter{tocdepth}{1}
% \tableofcontents
\clearpage

% \pagestyle{plain}

% \renewcommand{\chapter}{Lecture}


\section{arena}

\begin{tabular}{ p{0.3\linewidth} p{0.01\linewidth} p{0.6\linewidth}}
    \textbf{area}                 & : & Between $80m^2$ and $100m^2$. \\
    \textbf{style}                & : & Country standard. \\
    \textbf{decoration}           & : & According to the customs of the country. \\
    \textbf{minimum Composition}  & : & Living room, office, bedroom and kitchen.
\end{tabular}


\subsection{walls}

\begin{tabular}{ p{0.3\linewidth} p{0.01\linewidth} p{0.6\linewidth}}
    \textbf{Minimum height}       & : & 60 cm \\
    \textbf{Maximum height}       & : & It should not disrupt the public's view. \\
    \textbf{Decoration}           & : & The walls may contain poster or pictures. 
\end{tabular}

\subsection{Floor}

\begin{tabular}{ p{0.3\linewidth} p{0.01\linewidth} p{0.6\linewidth}}
    \textbf{Type}                 & : & With some grip to prevent robots from shaking. \\
    \textbf{Color/texture}        & : & Same color or texture of the wall \\
    \textbf{Steps}                & : & There should be no steps. \\
    \textbf{Carpet}               & : & There should be no carpet. 
\end{tabular}

\subsection{Doors}

\begin{tabular}{ p{0.3\linewidth} p{0.01\linewidth} p{0.6\linewidth}}
    \textbf{Amount}           & : & At least two doors. Entry and exit of the arena. The entrance door should open to inside the houve, as the output door should open to outside the house. \\
    \textbf{Knobs}            & : & The door handles should be rectangular preferably. Do not use round door handles. 
\end{tabular}

\subsection{Furniture}

\begin{tabular}{ p{0.3\linewidth} p{0.01\linewidth} p{0.6\linewidth}}
    \textbf{Type}       & : & Typical of the country. \\
\end{tabular}

\begin{description}
    \item [1] small dining table with two to four chairs.
    \item [1] sofa with two or three places.
    \item [2] chairs or armchairs.
    \item [1] Television hack.
    \item [1] television with remote control.
    \item [1] bookshelf (with some books inside).
    \item [1] a refrigerator (should not be less than 120 cm).
    \item [2] plant glass.
    \item [1] dishwasher.
    \item [1] bed.
    \item [1] Office table with chair.
    \item [2] trash bins with 20 cm (minimum) diameter at the opening. They must have removable lids with rigid straps.
\end{description}

\section{Arena objects}

\subsection{Table utensils}
\begin{description}
    \item [1] Set of dishes (minimum 3 items).
    \item [1] Set of bowls (minimum 3 items).
    \item [1] Set of cups (minimo 3 items).
    \item [1] Napkin package.
    \item [1] Set of cutlery (fork, knife and spoon).
    \item [10] plastic trash bags (medium size).
    \item [4] Shopping bags: Light and with rigid straps.
    \item [2] trays: tray or basket, intended for handling with both hands.
    \item [1] Object that I can put liquid (for example, jar).
    \item [1] heavy object: weight between 1.0kg and 1.5kg (eg water bottle).
    \item [1] Tiny object: light, not larger than 5 $cm^3$ (eg tea bag).
    \item [1] Fragile object: easy to break (eg egg).
    \item [1] A deformable object: flexible that can appear in different forms (eg fabric).
\end{description}

\subsection{Objects for recognition and manipulation}

The objects are presented to competitors with kits.

\begin{tabular}{ p{0.3\linewidth} p{0.01\linewidth} p{0.6\linewidth}}
    \textbf{Manipulação}   & : & todos os objetos devem ser pasiveis de serem manipulados com apenas uma mão e pesar no máximo 1.5KG. \\
    \textbf{Quantidade}     & : & A quantidade de kits é definida pelo número de equipes participantes. 1 kit para a arena e 1 kit para cada 3 equipes participantes. \\
    \textbf{Definição}       & : & Os Itens são definidos durante a competição (Warm Up). \\
    \textbf{Itens}                & : & Os kits devem ter entre 15 e 20 itens. Podem conter, por exemplo, os elementos a seguir. 
\end{tabular}

\resizebox{\textwidth}{!}{
    \begin{tabular}{|c|c|c|c|c|c|}
        \hline
        Drinks & Cleaning supplies & Pantry items & Fruits & Snacks & Cutlery\\
        \hline
        Coconut Water   & Cleaner  & Corn Flakes       & Apple   & Cookie       & Spoon \\
        Coke                      & Cloth       & English Sauce	 & Banana & Pringles   & Fork\\
        Guarana               & Sponge     & Mustard              & Orange & ~                   & Knife\\
        Ice Tea	               & ~                 & Tomato Sauce   & Kiwi      & ~                   & Plate\\ 
        Tonic                   & ~                 & Tuna Can            & ~             & ~                   & Bowl\\
        Water                   & ~                 & Corn Flakes     & ~             & ~                   & Mug\\
    \hline
    \end{tabular}
}

NOTE: A representative of the local organization must accompany the category coordinator to a local supermarket to overcome the items.

\section{Outros materiais}

\begin{description}
    \item [2] Cronômetros.
    \item [8] Suportes para poster.
    \item [1] Sistema de som com pelo menos uma lapela para ligar no robô.    
    \item [1] Monitor para apresentação de placar e informaçõe para o público.
\end{description}



\end{document}
