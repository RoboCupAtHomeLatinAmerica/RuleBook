%% %%%%%%%%%%%%%%%%%%%%%%%%%%%%%%%%%%%%%%%%%%%%%%%%%%%%%%%%%%%%%%%%%%%%%%%%%%%
%%
%%    author(s): RoboCupAtHome Technical Committee(s)
%%  description: Introduction - Leagues
%%
%% %%%%%%%%%%%%%%%%%%%%%%%%%%%%%%%%%%%%%%%%%%%%%%%%%%%%%%%%%%%%%%%%%%%%%%%%%%%
\section{Leagues}
\label{sec:introduction:leagues}

\AtHome{} is divided in three leagues. Two are \SPLs{} where each team uses the same robot platform and an \OPL{} where teams are free to choose their robot. The official leagues and their names are:
\begin{itemize}
  \item \DSPL{}
  \item \SSPL{}
  \item \OPL{}
\end{itemize}

\noindent All leagues share the same set of rules. The \DSPL{} uses the \HSR{} platform shown in figure \ref{fig:toyotaHSR} and the \SSPL{} uses the \PEPPER{} platform shown in figure \ref{fig:softbank-pepper}.

\begin{minipage}{0.5\textwidth}
	\begin{figure}[H]
		\begin{center}
			\includegraphics[height=0.6\textwidth]{images/toyota_hsr.png}
			\caption{Toyota HSR}
			\label{fig:toyotaHSR}
		\end{center}
	\end{figure}
\end{minipage}
\begin{minipage}{0.5\textwidth}
	\begin{figure}[H]
		\begin{center}
			\includegraphics[height=0.6\textwidth]{images/softbank_pepper.png}
			\caption{Softbank / Aldebaran Pepper}
			\label{fig:softbank-pepper}
		\end{center}
	\end{figure}
\end{minipage}