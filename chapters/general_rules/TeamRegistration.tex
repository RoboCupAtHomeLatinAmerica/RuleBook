%%%%%%%%%%%%%%%%%%%%%%%%%%%%%%%%%%%%%%%%%%%%%%%%%%%%%%%%%
\section{Team Registration and Qualification}
\label{sec:rules:particpation}

In order to participate, a team must answer the \CFP{} announced on the \AtHome{} mailing list by sending in their \Application{}. Then, they need to be selected in the \Qualification{} phase and finally, complete their \Registration{}.


\subsection{Application}
\label{sec:rules:application}

An application consists of a \TeamVideo{}, \TeamWebsite{}, and \TDP{}.

\paragraph{Team Video}
\label{sec:rules:application:video}
The \TeamVideo{} should show that the team has a running robot platform that is able to, at least partly, solve the tests in the last rulebook. Therefore, the video should focus on abilities required in \SONE{} and also include some skills required in \STWO{}. The video should be self-explanatory and designed for a general audience. Any editing, e.g. speed up, needs to be indicated. The video should not exceed \SI{10}{\minute} and needs to be publicly uploaded.


\paragraph{Team Website}
\label{sec:rules:application:website}
The \TeamWebsite{} should be designed for a broad audience. Therefore, it should include scientific material, but also interesting media. The default language of the website needs to be English.


\paragraph{Team Description Paper}
\label{sec:rules:application:tdp}
The \TDP{} is describing the team's main research, including the scientific contributions, goals, scope, and results, as well as, describing the used hardware. It needs to be in Portuguese or English, up to eight pages long, and formatted according to the guidelines of the \RoboCup{} Symposium. 
The \TDP{} must be sent via google docs form\footnote{\url{https://docs.google.com/forms/d/1Y1h0QkSyrFyvKn9UYOAl_9-WXs6fUb9oHZFUcYsNA7k/edit?ts=626f1802}}.
An addendum as the extra page (after references) needs to include:
\begin{itemize}
	\item Team Name
	\item Contact Information
	\item Website Url
	\item Team Member Names
	\item Photo(s) of the Robot(s) (unless included previously) 
	\item List of External Devices (see~\ref{sec:rules:externaldevices})
	\item List of 3rd Party Software
\end{itemize}

\subsection{Qualification}
\label{sec:rules:qualification}
The \OC{} will select teams for \Qualification{}. The selections will mainly be based on:
\begin{itemize}
	\item The content on the \TeamWebsite{}, focusing on publications and open source resources
	\item Number and complexity of abilities shown in the \TeamVideo{}
	\item Scientific value, novelty and contributions in the \TDP{}

\end{itemize}
Secondary evaluation criteria are:
\begin{itemize}
	\item Performance in previous competitions
	\item Previous contributions to the \AtHome{} community
\end{itemize}


\subsection{Registration}
\label{sec:rules:registration}
Qualified teams can register at the \RoboCup\AtHome{} competition. In order to max out the number of participants, qualified teams \emph{must} contact the \OC{} to confirm (or cancel) participation.

Confirming implies that the team has sufficient resources to complete \Registration{} and attend the competition. Teams that fail to confirm their participation will be disqualified.

\subsection{Task Selection Announcement}
\label{rule:task-selection-announcement}

To facilitate the scheduling of test slots during the competition, each team is required to send a list of tasks that they intend to participate in.
The list should be sent to the OC two weeks before the competition and should include all tests that the team intends to attempt.
The selected tasks should be according to the number and type of test slots of the competition (see~\Cref{tbl:max-elegible-tests}).
Teams are not obliged to perform all tasks on the list and are free to opt-out of any of them during the competition; however, a team cannot participate in a task if it was not included in the notification, i.e. opting-in for a task during the competition is not possible.
Please note that this rule is only there so that the organization is easier, as there are more tasks than testing slots available.

\begin{table}[H]
\centering
\caption{Maximum number of elegible tests}
\label{tbl:max-elegible-tests}
\newcolumntype{C}[1]{>{\centering\arraybackslash}p{#1}}
\begin{tabularx}{0.6\linewidth}{X C{0.4\linewidth}}
	\toprule
	                  & Stage II\\
	\midrule
	\textsc{Housekeeper}  &      3                   \\
	\textsc{Party Host}   &      3                   \\
	\bottomrule
\end{tabularx}
\end{table}
